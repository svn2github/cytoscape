
\section{Foundation of Genetics}
\label{founda}

In this section, some of basic knowledges that are necessary to 
understand this paper are described.

\subsection{What are genes?}
Why is a child similar to his or her parents? This is because the child
receives the genes from his/her parents.

Genes are the factors that determine the hereditary characteristics of
organisms. A lot of information on what the organisms would be like
is written on genes. The organisms are formed according to those genes.
More precisely, proteins, which are the main factor that characterizes
organisms, are synthesized according to the information in genes.
The gene is, in precise, a region of the molecule called DNA. 
DNA exists in the cell and forms a chromosome.
In eucaryotes(explained in \ref{cell}.),
chromosomes exist in the nucleus of the cell.


\subsection{Cells}\label{cell}
All the organisms are made of a cell or cells. According to the type
of the cell, some organisms are classified into procaryotes, and some
are classified into eucaryotes.
The composition of a cell is different. Generally, procaryotes have
simpler cell compositions than that of eucaryotes.

\begin{figure}
\epsfile{file=cell_pro.ps,scale=0.8}\ \ \ \ \ \ \ 
\epsfile{file=cell_eu.ps,scale=0.8}
\caption{Procaryotic cell and Eucaryotic cell}
\end{figure}

A cell consists of four kinds of molecules, i.e., sugar, fatty acid, 
amino acid and nucleotide. 
Sugar molecules form polysaccharide and fatty acid molecules
form lipid. And the chain of nucleotides form nucleic acid.
And amino acids linked by peptide bonds form protein.
Nucleic acid and protein are very large molecule and they
play important roles for organisms in maintaining life and species. 

\begin{table}
\begin{center}
\begin{tabular}{|l|l|}
\hline
monomer & polymer\\
\hline
\hline
sugar & polysaccharide \\
fatty acid & lipid \\
nucleotide & nucleic acid \\
amino acid & protein \\
\hline
\end{tabular}

\end{center}
\caption{Basic molecules of a cell}
\end{table}

\subsection{Amino acids and proteins}

Most characteristic of organisms are determined 
by proteins. It forms tissue, catalyzes chemical activities in 
organisms, etc.. The function of a protein is determined by the kinds
of amino acids which the protein consists of. There are 20 kinds of 
amino acid. Each amino acid has its characteristics such as
hydrophilic/hydrophobic, acidic/basic, etc.. According to kinds of
amino acids and their characteristics, they form three dimensional
structures and each functions differently.
They may interact with other proteins.


\subsection{Nucleotides and nucleic acids}
Nucleotide consists of base, sugar, and phosphate. And there are 4
kinds of bases. Those are adenine, thymine/uracil, cytosine, and
guanine, which are abbreviated as A, T/U, C, G. When they are linked,
they form nucleic acid. Nucleic acid has direction:One end is
called 5' terminal and the other is called 3' terminal.
Nucleic acid can be expressed by its sequence of nucleotides such
as ``atcgatgcctga....'' by writing each nucleotides included in nucleic
acid from 5' terminal to 3' terminal.

There are two kinds of nucleic
acid. Those are DNA(deoxyribonucleic acid) and RNA(ribonucleic acid).
Information about what kind of proteins will be made is stored in DNA,
which includes gene region, and it is passed to descendants.
DNA chains are double stranded, and A pairs with T, and C pairs
with G. On the other hand, RNA is usually single stranded.
Nucleotide included in RNA can easily make pairs with other nucleotide
included in RNA itself. 
The structure that an RNA chain forms by self pairings is called a
secondary structure.
As RNA can form higher order structure, it can mediate some activities
in the cell, whereas DNA usually works only as a template.
There are 3 kinds of RNA and each has different functions.
Base T is used in DNA and base U is used instead in RNA.

\begin{table}
\begin{center}
\begin{tabular}{|l|l|}
\hline
name of nucleic acid & main function\\
\hline
\hline
DNA & carries information of genes\\
\hline
mRNA & copies of DNA and work as template in synthesis of protein\\
\hline
tRNA & links a group of three nucleotides(codon) with 
amino acids \\
\hline
rRNA & helps to synthesize proteins in ribosomes\\
\hline 
\end{tabular}
\end{center}
\caption{Type of nucleic acid}  
\end{table}

\subsection{Transcription and translation}
How the protein is made according to the information written on DNA is
described in this subsection.
DNA is not directly converted to protein.
First, an enzyme called polymerase reads nucleotides on DNA chain  
 and synthesizes a molecule called mRNA. This process
is called transcription.
And a molecule called ribosome reads nucleotides on mRNA and
synthesizes a protein according to nucleotides on mRNA. 
This process is called translation.

In the process of transcription, the nucleotide sequence written on 
DNA is copied to
mRNA. But A is transcribed into U, T is transcribed into A, 
C is transcribed into G, and G is transcribed into C. 

In the process of translation, protein is synthesized according to the 
information written on mRNA. A group of three nucleotides(called a codon
and there are \(4^{3}=64\) of them)
 corresponds to one of 20 kinds of amino acids. This correspondence is
mainly determined by the RNA molecule called tRNA. This correspondance
is almost the same for all organisms.

It is known that nucleotide pattern near transcription
initiation site is likely to be ``TATA'' and nucleotide pattern on
translation initiation site is usually ``AUG''.

% A ribosome, where the proteins are synthesized, consists of proteins and 
% an RNA molecule called rRNA(ribosomal RNA). 

\begin{figure}
\begin{center}
\epsfile{file=discr2gs2.ps,scale=0.8}
\end{center}
\caption{A pathway from DNA to protein}

\end{figure}