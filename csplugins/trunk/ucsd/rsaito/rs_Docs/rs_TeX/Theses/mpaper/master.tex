\documentstyle[12pt,epsf]{report}
\topmargin=-45pt
\oddsidemargin=0cm
\evensidemargin=0cm
\textheight=23.7cm
\textwidth=16cm

\def\thebibliography#1{\chapter*{References\markboth
 {References}{}}\list
 {[\arabic{enumi}]}{\settowidth\labelwidth{[#1]}\leftmargin\labelwidth
 \advance\leftmargin\labelsep
 \usecounter{enumi}}
 \def\newblock{\hskip .11em plus .33em minus .07em}
 \sloppy\clubpenalty4000\widowpenalty4000
 \sfcode`\.=1000\relax}
\let\endthebibliography=\endlist

% \def\thebibliography#1{\section*{References\markboth
%  {References}{}}\list
%  {[\arabic{enumi}]}{\settowidth\labelwidth{[#1]}\leftmargin\labelwidth
%  \advance\leftmargin\labelsep
%  \usecounter{enumi}}
%  \def\newblock{\hskip .11em plus .33em minus .07em}
%  \sloppy
%  \sfcode`\.=1000\relax}
% \let\endthebibliography=\endlist

\title{Computer Analyses of\\ Translation Initiation Sites of Genes}
\author{
 Tomita Laboratory ATG Project\\
 \begin{tabular}{|c|c|l|}
 \hline
  Name            & Student ID & Affiliation \\
 \hline
  Rintaro Saito   & 89531794 & Graduate School of Media and Governance \\
  Hidekazu Sasaki & 79401975 & Faculty of Policy Management \\
  Yuko Osada      & 79450924 & Faculty of Environmental Information \\
  Yukari Shimizu  & 79452137 & Faculty of Environmental Information \\
 \hline
\end{tabular}\\
% Takefuji Laboratory Neural Computing Group\\
Keio University Shonan Fujisawa Campus
}
\date{}
\begin{document}
\maketitle


\begin{Large}
\verb+       +Abstract of Master's Thesis

\vspace{1ex}
\verb+                       +Academic Year 1996\\
\end{Large}

\begin{LARGE}
\begin{center}
Computer Analyses of\\ Translation Initiation Sites of Genes
\end{center}
\end{LARGE}

\vspace{4ex}
\begin{small}
Gene is one of the main factors that determine how the organism will
be like. The information of gene is written on the molecule called DNA
in the cell. But organisms use only part of DNA as genes. The exact
mechanism of distinguishing gene regions in DNA is still unknown.
Previous research showed that there is some remarkable patterns in the 
boundary region of genes on DNA. 
In this paper, we focused on one of the boundaries called translation
initiation site. 
By introducing computer programs that we developed, comprehensive analyses
of sequencing results 
 in GenBank database(database for genetic sequences) were conducted.
This analyzing method is an original and new one.
We have discovered tendencies that would be important. For
example, there  
is general tendency to have fewer AUG trinucleotides around start
codons. And higher organisms tend to keep the upstream AUG trinucleotides
farther from start codons.  \\ 

% for translation  initiation.
% Translation is a process that ribosomes synthesize protein by scanning
% a mRNA from left to right.  Although ribosome initiates translation at
% the first found AUG trinucleotide in eucaryotes most of the time,
% it sometimes skips
% and ignores the first AUG and initiates translation at the second or
% third AUG trinucleotide.  On the other hand in procaryotes, 
% Shine-Dalgarno sequence
% seems to play a important role for translation initiation, but this
% sequence is not completely conserved. And there are many regions that have
% sequences very similar to Shine-Dalgarno sequences and does not serve as
% translation initiation site. 
% After all, the exact mechanism 
% of selecting AUG for translation initiation is not known.  
% By using GenBank database(database for nucleotide sequences) and 
% computer programs, comprehensive analyses were conducted.
% We have discovered tendencies that may be important.
% for translation  initiation.
% Some results show the new aspects of characteristics of translation 
% initiation and its site.
\end{small}

\begin{small}
This work was done by Tomita Laboratory ATG Projects.
Main roles of each members are briefly described below. 

\begin{description}
\item[Academic adviser:] M.Tomita
\item[Project leader:] R.Saito
\item[Survey:] R.Saito
\item[Construction of main analysis program] R.Saito
\item[Analyses of profile around start codons] 
H.Sasaki(program and analyses of
eucaryotes), \\Y.Osada(procaryotes), Y.Shimizu(mitochondria), R.Saito(adviser) 
\item[Analyses of entropy values around start codons] Y.Osada
\item[Analyses of GC/AT contents in 5'UTR] R.Saito
\item[Analyses of frequencies of AUG trinucleotides around start
codons] R.Saito
\item[Analyses of rate of leaky scanning and distance between two
AUGs] R.Saito,H.Sasaki
\item[Analyses of tendencies when two AUGs are close to each other]
R.Saito, H.Sasaki
\item[Analyses of relationship between initiation mechanism and
evolution]
R.Saito
\item[Analyses of frequencies of stop codons located downstream of
skipped AUGs] H.Sasaki

\end{description}
\end{small}

\vspace{2ex}
\noindent
Key Words
\begin{small}
\begin{center}
1.Gene\ \ \  2.Translation Initiation \ \ \ 3.Computer Analyses \ \ \ 4.Start Codon\ \ \ 5.Leaky Scanning

\end{center}
\end{small}
\vspace{2ex}

\begin{center}
{\large Keio University Graduate School of Media and Governance}
\end{center}

\begin{large}
\verb+                                             +Rintaro Saito
\end{large}

\tableofcontents


\chapter{Overview}
\label{overv}

\section{Introduction}

Recently, a large amount of data on a variety of biological
experiments can be obtained freely through 
the internet. And as for species like {\it Mycoplasma
genitalium}\cite{label27},  {\it Haemophilas influenzae}, and yeasts, 
their whole DNA sequence data have already been read and we can easily 
obtain the data.
Obviously, amount of biological data such as DNA sequence data is increasing
drastically. On the other hand, amount of knowledge obtained from
these data is not increasing so fast.\\

\begin{center}
\epsfile{file=gengrow.ps,scale=0.6}\\
Figure: Increasing amount of biological data
\end{center}

To understand organisms, it is very important to 
find what meaning is hidden in these data. As the size of data is extremely
large, computer is very important tool for the fast analysis and knowledge
discovery. This new scientific field in which computer analyses of
biological data are done is called $Bio-informatics$. 

Biological experiments may investigate what kind of phenomenon 
occurs and how the phenomenon occurs in the specific cases. At the
same time, investigation
of organism by biological experiments has some disadvantages.

\begin{itemize}
\item Biological experiments are time consuming works, which often spend much
time to get single result.
When we want to know the universal law for a phenomenon, much data is needed.
But biological experiments take too much time.

\item There is the case where it is very difficult to see the cell and
its component at  
the molecular level. For example, protein is very important component for 
the cell, but it is very difficult to see its structure.
% Figure above shows number of entries for 
% GenBank(database for nucleic acid sequences)
% and PDB(database for structures of proteins) increasing each year. Although
% number of nucleic acid registered is increasing very rapidly, 
% number of structures
% of proteins registered is not increasing so fast.
Furthermore, looking the interaction of the component at the molecular level
is much more difficult.

\end{itemize}
 
$Bio-informatics$ investigates organisms from the view of information. 
By analyzing large amount of data in databases from various aspects, 
universal law for a phenomenon may be discovered. And by analyzing 
data which seems to be important for a specific interaction at the molecular
level, we can infer how the interaction occurs. Those
cannot be done by biological experiments. However, biological experiments are
important because it is only the way to directly observe the phenomenon.

So biological research must be done from two approaches to understand
 the organism.

\begin{itemize}
\item Approach from the experiments
\item Approach from the data analyses
\end{itemize}

In this paper, we approach to biology from the data analyses. The
topic is translation initiation.
The goal of this research is to find how the organisms recognize specific
DNA sequence as genes. Gene is information about how the organisms would
be like. This information is written on molecule called DNA in the cell.
But only part of DNA is used as genes. Then the question ``How do the 
organisms distinguish gene region of DNA?'' arises. The exact mechanism
is still unknown. If we discover new mechanisms, it must be the contribution
not only to biology, but also to biotechnology and medical.

We are conducting comprehensive computer analyses around boundaries
of gene regions. If the remarkable patterns are found, this may be 
influencing the distinction of gene region. 

In this paper, we focus on boundary called translation initiation site
and discuss the tendencies we discovered from this site.


\section{Purpose}

The goal of this research is to understand how the ribosomes 
determine the translation initiation site. In this paper, we use
comprehensive computer analyses, and try to discover tendencies that
 may be important for the selection of translation initiation site by 
ribosomes and infer the mechanism of them.

First, we conduct profile analyses and entropy calculations around
start codons to verify the hypotheses of
M.Kozak(eucaryotes)\cite{label3}, J.Shine and 
L.Dalgarno(procaryotes)\cite{label7}, and other researchers on translation
initiation. And further analyses are conducted to find remarkable tendencies.

Secondly, tendencies including frequencies of AUG trinucleotides
around start codons are invenstigated to support our original
hypothesis that ribosomes confuse if two AUGs are located close to
each other.

Thirdly, analyses of stop codons are conducted to investigate the
importance of reinitiation for leaky scanning, whose phenomenon is
investigated by M.Kozak\cite{label18} and other researchers. 

\section{Originality}

In the field of $bio-informatics$, there are many researches for making
the algorithm to analyze data. And there are many researches for analyzing
a specific biological substance or phenomenon such as analyses of protein 
function, analyses of third dimensional structure of protein, analyses
of metabolic pathway, analyses of evolution at the molecular model,etc. 
Databases are used to help their work.

However, there are few researches which are done by analyzing whole database
itself by creating original programs to find tendencies and infer the
biological mechanisms.
We have done it for translation initiation site from various points of view.


% \section{Significance of this Research}
% By analyzing translation initiation site from database, we may
% discover the mechanism of translation initiation that is not known
% so far. In the long term, this will make contribution not only to
% biology, but also to medical, and to biotechnology. 


\section{Results Expected}

By conducting comprehensive computer analyses of large amount of sequences
in database, we can expect the following results.

\begin{itemize}

\item We can get data that support the already known hypotheses on
translation initiation.
\item We can get data that is consistent with our original hypotheses
on translation initiation. 
\item We can discover tendencies by analyzing database from various
view points. From these discoveries, we can make new hypotheses.

\end{itemize}

% we can discover universal rule of ribosomes in determining
% the translation initiation site.

\section{Composition of This Paper}

Chapter \ref{matmeth} describes materials and methods for this research.
Chapter \ref{backgr} explains the foundation of genetics which is
needed to understand this paper(section \ref{founda}) and mechanism of
translation initiation discovered so far by other researches(section
\ref{surv}).  
Chapter \ref{resu} introduces our research and its results. 
In this chapter, section \ref{prof_ent} is mainly data analyses to
confirm the known 
hypotheses and section \ref{lowfreq} and section \ref{reinitia} are
mainly data analyses to confirm our original hypotheses.
Chapter \ref{concl} concludes our paper with future prospects.




\chapter{Materials and Methods}

In this chapter, original method used for the analyses is described.

\label{matmeth}

\section{Flow}

We make some hypothesis on translation initiation and its sites. Then
we access database to get data and calculate, using computer programs
we develop, to check whether
the results are consistent with our hypothesis or not.
Researches are done by the flow shown in figure \ref{res_flow}.

\begin{itemize}
\item Decide topic\\
Decide what topic to investigate. In this paper, the whole topic is
translation initiation.

\item Survey about topic\\
Survey the researches that are done on the decided topic.

\item Make hypotheses\\
From the investigation done by other researchers so far,
make hypotheses on the topic.

\item Conduct data analyses\\
Make analyses using database to investigate whether the hypotheses hold
or not. To analyze the database, write computer programs.

\item Analyze results\\
Analyze the output from the computer programs.

\item Discussion\\
Examine whether the output matches to the hypotheses or not.
Depending on the results, further analyses will be done.

\end{itemize}

\begin{figure}
\begin{picture}(400,220)

\put(100,210){\framebox(150,10){Decide topic}}
\put(175,205){\vector(0,-1){20}}

\put(100,170){\framebox(150,10){Survey about topic}}
\put(175,165){\vector(0,-1){20}}

\put(100,130){\framebox(150,10){Make hypotheses}}
\put(175,125){\vector(0,-1){20}}

\put(100,90){\framebox(150,10){Conduct data analyses}}
\put(175,85){\vector(0,-1){20}}

\put(100,50){\framebox(150,10){Analyze results}}
\put(175,45){\vector(0,-1){20}}

\put(255,15){\line(1,0){100}}
\put(355,15){\line(0,1){160}}
\put(355,175){\vector(-1,0){100}}
\put(355,135){\vector(-1,0){100}}

\put(100,10){\framebox(150,10){Discussion}}

\end{picture}
\caption{Flow of the research}
\label{res_flow}
\end{figure}

\section{Method for data analyses}\label{method_3}
For the data analyses, GenBank database release 94.0 can be used.
In this database, sequences are divided into some taxonomical groups.
Followings are the files we used.

\begin{description}
\item[gbpri.seq] Primate sequence entries
\item[gbrod.seq] Rodent sequence entries.
\item[gbmam.seq] Other mammalian sequence entries.
\item[gbvrt.seq] Other vertebrate sequence entries.
\item[gbinv.seq] Invertebrate sequence entries.
\item[gbbct.seq] Bacterial sequence entries.
\end{description}

Those files are installed as UNIX files and accessible from C
language. Thus for the analyses, we created original C programs to get
specific data and analyze them. 

\begin{figure}
\begin{picture}(400,30)
\put(90,0){\framebox{\shortstack[c]{GenBank\\ database}}} 
\put(190,5){\framebox{Original C Programs}}
\put(350,5){\framebox{Results}}

\put(150,9){\vector(1,0){30}}
\put(305,9){\vector(1,0){30}}
\end{picture}
\caption{Method for data analysis}
\end{figure}

We used mRNA data for the analysis, because mRNA is involved in
translation much more than DNA is. But as mRNA data for bacteria
are few, we used DNA data instead for bacteria.
Also when we calculate average distances between start codon and
nearest upstream AUG trinucleotides, we used DNA for the precise
analyses because there are more DNA data than mRNA data.

And following data are excluded from our analyses.
\begin{itemize}
\item Pseudo genes. Pseudo genes are not translated in vivo. Computer
program eliminates data in the entry which  has keyword
``pseudo'' either in DEFINITION or in CDS field.
\item Immunoglobulin and receptor sequences. They have special translation
features. Computer program eliminates data in the entry which has keyword
``immuno'',''receptor'' or ``variabl'' in DEFINITION field. 
\item Mitochondrial sequences except for the mitochondria analyses.
Mitochondria have special translation initiation features. Computer
program eliminates data in the entry which has keyword ``mitochond''
in DEFINITION field. 
\item Partial sequences. Computer program eliminates data in the entry 
which has keyword ``partial'' in either DEFINITION field or CDS field.
Also it eliminates entries with keyword ``exon'' in DEFINITION field. 
\item Sequences that have introns in 5'UTR. Computer program
recognizes the location of intron by the keyword ``intron'' and the
location following to that keyword.
\item Putative sequences. Computer program eliminates keyword
``/note="putative"'' in CDS field.
\item Sequences that have non-AUG start codon in eucaryotes. Since
eucaryotes almost always initiate translation from AUG codon in vivo, we
only accept data with AUG start codon. 
\item AUG trinucleotide in the same frame as start codon in 5'UTR.
Start codons in the database are likely to be wrong when two AUGs are
in the same reading frame. Thus, for the precise analysis, we
eliminated those data.


\end{itemize}







\chapter{Background}
\label{backgr}

\section{Foundation of Genetics}
\label{founda}

In this section, some of basic knowledges that are necessary to 
understand this paper are described.

\subsection{What are genes?}
Why is a child similar to his or her parents? This is because the child
receives the genes from his/her parents.

Genes are the factors that determine the hereditary characteristics of
organisms. A lot of information on what the organisms would be like
is written on genes. The organisms are formed according to those genes.
More precisely, proteins, which are the main factor that characterizes
organisms, are synthesized according to the information in genes.
The gene is, in precise, a region of the molecule called DNA. 
DNA exists in the cell and forms a chromosome.
In eucaryotes(explained in \ref{cell}.),
chromosomes exist in the nucleus of the cell.


\subsection{Cells}\label{cell}
All the organisms are made of a cell or cells. According to the type
of the cell, some organisms are classified into procaryotes, and some
are classified into eucaryotes.
The composition of a cell is different. Generally, procaryotes have
simpler cell compositions than that of eucaryotes.

\begin{figure}
\epsfile{file=cell_pro.ps,scale=0.8}\ \ \ \ \ \ \ 
\epsfile{file=cell_eu.ps,scale=0.8}
\caption{Procaryotic cell and Eucaryotic cell}
\end{figure}

A cell consists of four kinds of molecules, i.e., sugar, fatty acid, 
amino acid and nucleotide. 
Sugar molecules form polysaccharide and fatty acid molecules
form lipid. And the chain of nucleotides form nucleic acid.
And amino acids linked by peptide bonds form protein.
Nucleic acid and protein are very large molecule and they
play important roles for organisms in maintaining life and species. 

\begin{table}
\begin{center}
\begin{tabular}{|l|l|}
\hline
monomer & polymer\\
\hline
\hline
sugar & polysaccharide \\
fatty acid & lipid \\
nucleotide & nucleic acid \\
amino acid & protein \\
\hline
\end{tabular}

\end{center}
\caption{Basic molecules of a cell}
\end{table}

\subsection{Amino acids and proteins}

Most characteristic of organisms are determined 
by proteins. It forms tissue, catalyzes chemical activities in 
organisms, etc.. The function of a protein is determined by the kinds
of amino acids which the protein consists of. There are 20 kinds of 
amino acid. Each amino acid has its characteristics such as
hydrophilic/hydrophobic, acidic/basic, etc.. According to kinds of
amino acids and their characteristics, they form three dimensional
structures and each functions differently.
They may interact with other proteins.


\subsection{Nucleotides and nucleic acids}
Nucleotide consists of base, sugar, and phosphate. And there are 4
kinds of bases. Those are adenine, thymine/uracil, cytosine, and
guanine, which are abbreviated as A, T/U, C, G. When they are linked,
they form nucleic acid. Nucleic acid has direction:One end is
called 5' terminal and the other is called 3' terminal.
Nucleic acid can be expressed by its sequence of nucleotides such
as ``atcgatgcctga....'' by writing each nucleotides included in nucleic
acid from 5' terminal to 3' terminal.

There are two kinds of nucleic
acid. Those are DNA(deoxyribonucleic acid) and RNA(ribonucleic acid).
Information about what kind of proteins will be made is stored in DNA,
which includes gene region, and it is passed to descendants.
DNA chains are double stranded, and A pairs with T, and C pairs
with G. On the other hand, RNA is usually single stranded.
Nucleotide included in RNA can easily make pairs with other nucleotide
included in RNA itself. 
The structure that an RNA chain forms by self pairings is called a
secondary structure.
As RNA can form higher order structure, it can mediate some activities
in the cell, whereas DNA usually works only as a template.
There are 3 kinds of RNA and each has different functions.
Base T is used in DNA and base U is used instead in RNA.

\begin{table}
\begin{center}
\begin{tabular}{|l|l|}
\hline
name of nucleic acid & main function\\
\hline
\hline
DNA & carries information of genes\\
\hline
mRNA & copies of DNA and work as template in synthesis of protein\\
\hline
tRNA & links a group of three nucleotides(codon) with 
amino acids \\
\hline
rRNA & helps to synthesize proteins in ribosomes\\
\hline 
\end{tabular}
\end{center}
\caption{Type of nucleic acid}  
\end{table}

\subsection{Transcription and translation}
How the protein is made according to the information written on DNA is
described in this subsection.
DNA is not directly converted to protein.
First, an enzyme called polymerase reads nucleotides on DNA chain  
 and synthesizes a molecule called mRNA. This process
is called transcription.
And a molecule called ribosome reads nucleotides on mRNA and
synthesizes a protein according to nucleotides on mRNA. 
This process is called translation.

In the process of transcription, the nucleotide sequence written on 
DNA is copied to
mRNA. But A is transcribed into U, T is transcribed into A, 
C is transcribed into G, and G is transcribed into C. 

In the process of translation, protein is synthesized according to the 
information written on mRNA. A group of three nucleotides(called a codon
and there are \(4^{3}=64\) of them)
 corresponds to one of 20 kinds of amino acids. This correspondence is
mainly determined by the RNA molecule called tRNA. This correspondance
is almost the same for all organisms.

It is known that nucleotide pattern near transcription
initiation site is likely to be ``TATA'' and nucleotide pattern on
translation initiation site is usually ``AUG''.

% A ribosome, where the proteins are synthesized, consists of proteins and 
% an RNA molecule called rRNA(ribosomal RNA). 

\begin{figure}
\begin{center}
\epsfile{file=discr2gs2.ps,scale=0.8}
\end{center}
\caption{A pathway from DNA to protein}

\end{figure}

% \section{Recent Advance in This Field}

Although bio-informatics is a new field of research, 
there are many field of research.

\subsection{Sequencing}

Sequencing is the reading of the nucleic acid or amino acid sequences.
By using method called PCR, many DNA sequence data are read.
There are some organisms whose whole genome data were read.
That are, for example, bacteria like {\it Mycoplasma Genitalium},
{\it Haemophilus influenzae}. Whole human genome will be read in
Human Genome Project which is in progress by many countries.
And whole {\it E.Coli} genome data will also be read.
Those data will be open to the public through
the internet. To determine where the genes are in the sequence,
biological experiments must be done. Gene finding algorithm may be
helpful if it can appropriately identify where the genes are, but
the prediction rate is about 90\%. It needs improvement.

\subsection{Signal Analyses}

In DNA sequences, there are many signals such as transcription start signal,
transcription termination signal, etc.


\subsection{Prediction of the function of proteins}

In bio-informatics,prediction is done by homology search.

\subsection{Investigation of evolutions from the molecular level}

By this research, we can know how the species evolved in molecular level
rather than fossil level. This can be improved by alignment algorithms.


\subsection{Prediction of secondary or tertialy structure of nucleic 
acid or protein}

RNA secondary structure prediction is done by Zuker.


\subsection{Simulation of biological systems}

Methabolic simulation is done by Peter Karp, Douglas Brutlag.

\section{Ultimate Goal of This Field}

Ultimate goal of this field is to completely understand life from 
biological data such as DNA sequences. If this goal is accomplished, then
we can cure disease by modifying DNA sequences. And we can change our body
to whatever we wish in the same way. Those are the contribution to medicine.
Also in the same way, we can produce many kinds of foods. That is 
the contribution to biotechnology.

But we always have to keep in mind what we should not do with those 
technologies.






\section{Advance in the Research of Translation Initiation}
\label{surv}
In this section, mechanisms of translation initiation discovered by 
other researches are described.
The process of translation initiation in eucaryotes is different from that
of procaryotes.
We first discuss the translation initiation process of procaryotes and
then eucaryotes.


\subsection{Translation initiation in procaryotes}

In procaryotes, a ribosome with tRNA which carries methionine binds to 
the specific region of mRNA and recognizes AUG codon nearby
and protein synthesis begins. In this process the main factors are
ribosome, tRNA with methionine(fMet-tRNA$^{\rm Metf}$), and mRNA. In addition, 
at least initiation factors and GTP molecule are required to ensure
the efficiency and fidelity of this process.

The descriptions of each factors are given\cite{label20}.


\subsubsection{fMet-tRNA$^{\rm Metf}$}
A translation starts with methionine, whose codon is usually AUG.
Initiator tRNA(tRNA$^{\rm Metf}$) with methionine(fMet-tRNA$^{\rm Metf}$)
recognizes AUG codon and translates it to 
methionine.
AUG codon in the middle of the coding region is translated by 
Met-tRNA$^{\rm Met}$ which
is used for elongation of protein. fMet-tRNA$^{\rm Metf}$, which is used for 
translation
initiation, is different from Met-tRNA$^{\rm Met}$ in some aspects.
One of these is the presence of three consecutive GC base pairs in the 
anticodon stem which is important for the formation of anti-codon loop
and binding to the ribosomal P-site\footnote{P-site is where the end of
elongating amino acid with tRNA(peptidyl-tRNA) is located, 
waiting for the next amino acid with tRNA(aminoacyl tRNA)}.
Another difference is that \(\alpha\)NH2 group of the methionine in
 fMet-tRNA$^{\rm Metf}$ is blocked by formylation and it cannot form a peptide
bond in here. It means that it cannot work as elongation tRNA.

\subsubsection{Ribosome}
Bacterial ribosome which is used for translation is 70S\footnote{S values 
indicate the rate of sedimentation in an ultracentrifuge}.
It consists of 30S small subunit and 50S large subunit.
Furthermore, 30S small subunit contains 16S rRNA and 21 proteins, and
50S large subunit contains 5S rRNA, 23S rRNA and 34 proteins.
Only after the small ribosomal subunit loaded with initiation factors finds
the start codon does the large subunit bind.


\subsubsection{mRNA}
Bacterial mRNAs are commonly polycistronic. That means that they encode
multiple proteins that are separatedly translated from the same mRNA
molecule. Sometimes coding regions overlap, but it may not effect the
fidelity of
translation. Sometimes coding regions overlap by one base, which 
will be like [UG\underline{A]UG}.

In bacterial mRNA, ribosome binding site and start codon play important
roles for translation initiation. Ribosome binding site is where the 30S
small subunit binds first on mRNA. This site contains purine\footnote{Adenine
and guanine} rich sequence which is called Shine-Dalgarno sequence
\cite{label7}. The 3' terminal of 16S rRNA in 30S subunit binds
to this sequence and helps 30S subunit to bind to mRNA.
This interaction makes a major contribution to the efficiency of initiation
and provides the bacterial cell with a simple way to regulate protein
synthesis. Many translational control mechanisms in procaryotes involve 
blocking the Shine-Dalgarno sequence, either by covering it with a bound
protein or by incorporating it into a base-paired region in the mRNA
molecule. 

Most {\it E.Coli} mRNAs have Shine-Dalgarno sequences(except mRNA for dnaG
primase).
The sequence of 3' terminal of 16S rRNA in {\it E.Coli} is
"....acctgcggttggatcacctcctta". Shine-Dalgarno sequence should be
complement to this sequence, i.e., "taaggaggtgatccaaccgcaggt...." .
However, this sequence is not usually conserved, and only 3 to 9
nucleotides will make pairs with the 16S rRNA 3'terminal sequence.
If the length of complementarity is too long, ribosome will bind
to mRNA too tightly, disturbing the migration of the ribosome.
There are homology between {\it E.Coli} 16S rRNA and that of other bacteria.

Schneider et al. has analyzed translation initiation sites
in terms of information
theory\cite{label11}. And Barrick et al. has quantitatively analyzed 
Shine-Dalgarno sequence by the experiments\cite{label28}.
There is a possibility that 16S rRNA binds to
multiple regions of mRNA. This may be upstream of AUG, the spacer region 
between Shine-Dalgarno sequence and AUG, or downstream of AUG\cite{label10}.
But for downstream of AUG, the position +16 has been determined by experiments
as being the limit of the downstream region in the mRNA which interacts 
strongly with the ribosome.
I.G.Ivanov et al. has found a second putative
mRNA binding site on {\it E.Coli} 16S rRNA\cite{label23}.


 Distance between the Shine-Dalgarno sequence and
the start codon is known to affect efficiency of translation
significantly. 
Optimal distance between the Shine-Dalgarno sequence and the start codon was
estimated to be 4 to 9 base pairs\cite{label9}.
Three examples are listed below. The Shine-Dalgarno sequence and functional
initiator codon are shown in larger type, and the underlined AUG and
GUG triplets are not functional, presumably because they lie too close
to the Shine-Dalgarno sequence\cite{label22}.


\begin{description}
\item[{\it E.Coli} trpC]  {\large GAGG}GUA\underline{AUG}{\large AUG}
\item[coliphage QB polymerase]
{\large UAAGG\underline{A}}\underline{UG}AA\underline{AUG}C{\large AUG} 
\item[{\it E.Coli} lac I] 
{\large G\underline{GUG}}\underline{GUG}A\underline{AU{\large G}}{\large UG}
\end{description}




Start codon is AUG in most of the time, but sometimes GUG(8\%) and UUG(1\%)
work as start codon. In most cases, changing the rare initiation triplet
into the more common AUG will cause the increase of expression and this
may disturb the normal mechanism of gene expression. Thus one of the reasons
that the bacteria use start codon like GUG or UUG is that they may be
appropriate for the expression control of some specific genes. Notice that
when GUG is read as a start codon, it is translated to formyl-methionine.
If we change normal AUG to GUG, translation may not occur.


AUG triplets preceded by appropriately
spaced Shine-Dalgarno-like sequences appear randomly throughout the
{\it E.Coli} genome, which does not function as translation initiation site.
Thus, besides Shine-Dalgarno sequence and start codon and their distance,
there are some factors that affect translations. The important one is 
a secondary structure. It may separate or hide Shine-Dalgarno sequence
and/or AUG triplet.
Thus, an access of ribosome to non-initiation site of mRNA is restricted by the
secondary structure of mRNA. But a good Shine-Dalgarno complementarity
provides the ribosome with an increased affinity for its binding site,
and thereby enhances its ability to compete against the secondary 
structure\cite{label26}.

Another factors may be some consensus sequences. That are sequences 
preceding Shine-Dalgarno sequence,
sequences between Shine-Dalgarno sequence and start codon, and sequence
following the start codon. 
\begin{enumerate}
\item Except for ribosome binding sites, there are not many  G's and there are
many A's and U's.
\item position -3 is likely to be A and position +4 through +7 is
likely to be GCUA or AAAA. In one experiment, which changed AAA to AAG,
the translational efficiency reduced by 70\%\cite{label25}.
\item When the sequences preceding Shine-Dalgarno sequence is eliminated,
sometimes translation does not occur.

\end{enumerate}



 
\subsubsection{Initiation factors}
The initiation process is complicated, involving a number of steps catalyzed
by proteins called initiation factors, many of which are themselves
composed of several polypeptide chains. In {\it E.Coli}, three initiation 
factors namely IF1, IF2 and IF3 seem to play important roles for translation 
initiation. 30S subunit has a single high-affinity site for each factor.
But IF1 and IF3 show little affinity for 50S subunit and for 70S monomers.
In fact, they are ejected from the 30S subunit when 30S and 50S subunit
associate. But IF2 has fairly high affinity for 50S subunit and 70S monomers.

The main functions of these initiation factors are to control associations
and disassociations of codons and anti-codons of tRNA in P-site of the
ribosome and, ultimately, to influence which and how many 30S initiation 
complexes enter the elongation cycle after association with the 50S subunit.

Overview of functions of each initiation factors are listed below.
\begin{description}

\item[IF1] Main functions are obscure. 
It may be involved in some stabilization.

\item[IF2] It binds to 30S subunits. It contains a binding site for fMet-tRNA.
      It effects the formation of initiation complexes(which is an association
of  ribosome, mRNA, and some other factors)kinetically. Upon subunit 
interactions, it interacts with 50S subunit, activating enzyme for GTP.
It positions fMet-tRNA in P-site.

\item[IF3] It binds to 30S subunits and is ejected upon subunit association.
  When it is binding to 30S subunit, 50S subunit cannot associate. It is
  needed when 30S subunit binds to initiation site of mRNA. But it seems that
  it is not involved in the decision of translation start site.
  IF3 crosslinks to 16S rRNA in two regions in one time.\\

\end{description}
These three initiation factors do not share any structural homology
and appear to be evolutionarily conserved. For example, IF1 of {\it E.Coli}
shares 69\% of identical residues with IF1 of {\it B.subtilis}, while IF3
of {\it E.Coli} is 50\% identical with IF3 of {\it B.stearothermophilus}.

\begin{figure}
\begin{center}
\epsfile{file=transbct.eps,scale=0.8}
\end{center}
\caption{Simplified mechanistic model of translation initiation in
procaryotes}
\end{figure}



\subsection{Translation initiation in eucaryotes}
In eucaryotes, small subunit of ribosome first binds to the 5' end of
mRNA and then moves downward to the 3' direction by the hydrolysis of 
ATP, searching
for translation initiation site instead of
directly forming complexes around start codon like bacteria. 
In this process, the main factors are ribosome, tRNA with 
methionine(Met-tRNA$^{\rm Meti}$), and mRNA. In addition, many initiation factors 
are needed. It is far more than that of procaryotes.

\subsubsection{Met-tRNA$^{\rm Meti}$}
Like bacteria, the translation start with methionine.
In case of eucaryotes, tRNA$^{\rm Meti}$ is used for translation initiation.
It binds to methionine to form Met-tRNA$^{\rm Meti}$. Unlike bacteria,
this methionine is not formylated.

\subsubsection{Ribosome}
Eucaryotic ribosome which is used for translation is 80S. It consists
of 40S small subunit and 60S large subunit. Furthermore,
40S small subunit contains 18S rRNA and about 30 proteins,
and 60S large subunit contains 28S rRNA, 5.8S rRNA, 5S rRNA and
about 40 proteins. It is 40S subunit that binds to mRNA to scan for
translation initiation site. There are homology between 18S rRNA and
16S rRNA of {\it E.Coli}.

\subsubsection{mRNA}
Eucaryotic mRNA do not have Shine-Dalgarno sequences. Instead,
it has CAP structure. The CAP structure is a structure in
which phosphoric acid is attached to 5' terminal of mRNA
and it plays an important role for translation initiation. 
It is recognized by 40S subunit and 40S subunit binds to it.
5'UTR(UnTranslated Region) of vertebrate mRNA is usually GC rich.
It helps mRNA to form secondary structure by GC base pairings.
Secondary structure may disturb translation as it will be
described later. This means that expression of gene is controlled 
at the level of translation by this feature. Those mRNAs that are 
especially GC rich in 5'UTR are likely to be undertranslated when the general 
translational capacity declines. The translation of most cellular 
mRNAs is, in fact, inhibited by serum deprivation, or heat shock, or
virus infection. Actual GC content in 5'UTR partly depends on the kind of
protein that the mRNA makes. But most mRNAs have GC content higher
than 50\%. 

Start codon is AUG most of the time, but some experiment showed
that translation can be initiated from non-AUG codons, such as
ACG, CUG, or GUG\cite{label6}\cite{label29}.


\subsubsection{Initiation factors}
One of the main initiation factors in eucaryotes is eIF and there are
many kinds of eIF. Nine initiation factors of reticulocyte and its
functions are listed in table \ref{eif}.
\begin{table}
\begin{tabular}{|c|l|}
\hline
factor & ~~~~~~~~~~~~~~~~~~function\\
\hline
eIF3 & binds to mRNA\\
CBPI,CBPII & help ribosome to bind to CAP of mRNA and break secondary structure \\
eIF1,eIF4B & help binding of mRNA\\
eIF4A & helps binding of mRNA and binds to ATP\\
eIF6 & disturbs associations of 40S and 60S ribosomal subunits\\ 
eIF5 & separates eIF2 from eIF3\\
eIF4C & binds to 60S ribosomal subunit\\
eIF2 & binds to Met-tRNA$^{\rm Meti}$\\
eIF4D& unknown \\
\hline
\end{tabular}
\caption{Initiation factors in eucaryotes}
\label{eif}
\end{table}
Hannig identified yeast GCD10 gene as the structural gene for the yeast
eIF3 and analysis of mutant phenotypes has opened the door to the genetic
dissection of the eIF3 protein complex\cite{label24}.

\begin{figure}
\begin{center}
\epsfile{file=transeuc.eps,scale=0.8}
\end{center}
\caption{Simplified initiation cycle on ribosomes in eucaryotes}
\end{figure}

\subsubsection{Decision of translation initiation site}
In the "scanning model", ribosomal 40S subunit first binds to the CAP
structure and then moves downward to the 3' direction.  It is proved
by experiments that ribosome can not go back to 5'
direction\cite{label1}. Most of the time, ribosome uses the first
found AUG as a starting AUG for translation initiation.  However, it
sometimes passes this first AUG and looks for another AUG for
translation initiation.  This process is called "leaky
scanning"(figure \ref{dscr4}).

\begin{figure}
\begin{center}
\epsfile{file=dscr4.ps,hscale=0.85,vscale=0.75}
\end{center}
\caption{Leaky Scanning}
\label{dscr4}
\end{figure}

We can think of four possible causes in order for the leaky scanning to
occur.

\begin{itemize}
\item Nucleotide sequences around the AUG are not suitable for
initiation. \\
The preferred sequence around a starting AUG in
vertebrates, for example, is gccGCCACC[AUG]G\cite{label3}.

Position -3\footnote{Traditionally 
position +1 is defined as A-residue of the AUG initiation codon. The adjacent
nucleotide 5' with respect to this A-residue is defined as position -1
 Ex.C$^{-2}$
C$^{-1}$$A^{+1}T^{+2}G^{+3}$G$^{+4}$. But in this paper,
sometimes the position where A-residue of start ATG is located is defined 0.}
is often adenine or guanine\cite{label4};AUGs are likely to be skipped
if its -3 position is cytosine or thymine. But if guanine is
located in position +4 in this case, it may have some effect to
prevent the leaky scanning. 

\item The AUG trinucleotide is followed shortly by a stop codon.\\
Earlier studies showed that
ribosomes can reinitiate the translation from the second AUG 
if the first AUG is followed shortly by a stop codon in frame. Initiation 
factors dissociate from the 40S ribosomal subunit upon the addition of 
a 60S ribosomal subunit and initiation of peptide bond formation. But
if the 40S reaches stop codon right after the translation initiation, 
initiation factors are still on the 40S and it has an ability to
reinitiate translation\cite{label4}. The
efficiency of reinitiation increases when stop codons are located far
upstream of the start codon\cite{label18}.

\item Distance between the CAP structure and the AUG is too
short.\\  It has been shown that the leaky scanning is likely to occur
when the distance between the CAP structure and the AUG trinucleotide is too
short\cite{label19}.  
Some organisms take advantage of this characteristics to produce
alternative proteins from a single gene by alternating
the location of the CAP structure (i.e., transcription initiation site),
resulting alternating selection of a start codon.
\cite{label5}.

\item Secondary structure of mRNA affects translation initiation.\\
Secondary structures such as hair pin loops in mRNA may disturb the 
recognition of starting AUG\cite{label21}. Advancing on mRNA, 
40S subunit breaks the secondary structure of mRNA if the stability
of the secondary structure is below \(\Delta\)G = -30\footnote{Free
energy}. But if it has 
more stability, advancing subunit will slow down or stop, which may
disturb the translation initiation.
But in some cases, it may enhance the recognition of the 
starting AUG\cite{label2}. For example, when the bases around AUG
in mRNA are not in a favorable context, subunit is likely to pass by. 
But if the mRNA has hairpin structure downstream of this AUG triplet,
migrating ribosome may slow down, having more time to recognize AUG
as a start codon.
\end{itemize}

Note that a few eucaryotic mRNAs initiate translation by the different
mechanism 
rather than scanning. These mRNAs contain complex nucleotide
sequences called internal ribosome entry sites. Ribosomes bind here
without the help of CAP and start translation at the next AUG codon
downstream\cite{label0}.  

\subsection{Translation initiation in mitochondria}
Mitochondria are membrane-bounded organelles(organs that have specific
functions in the cell) that carry out oxidative phosphorylation and
produce most of the ATP in eucaryotic cells. They have their own
genetic systems;They have their own DNA and produce some proteins
of their own. 

Translation often starts from the site close to the transcription 
initiation site. 
For the start codon of mitochondria, AUG is used frequently. In addition, AUA
and AUU are also used.  






\chapter{Result of This Research}

We conducted comprehensive computer analyses on translation initiation 
site by using our method described in chapter \ref{matmeth}.
In section \ref{prof_ent}, we conduct analyses of nucleotide
distributions around start codons to verify the results of previous
research and to discover new tendencies.
In section \ref{lowfreq}, we conduct comprehensive analyses from
various view points to examine our hypothesis that  AUG trinucleotides
near start codons confuse ribosomes. In section \ref{reinitia}, we
focus on stop codons located downstream of skipped AUGs to investigate the
significance of those codons for leaky scannings.

\label{resu}
\section{Profile and Entropy around Start Codons}
\label{prof_ent}

The purpose of this section is to analyze tendencies of nucleotide
distributions around start codons to find whether the tendencies are
consistent with previous research and to discover remarkable tendencies.
In previous study, Kozak conducted similar analyses with
699 vertebrate mRNA sequences\cite{label3}.
But our sequence data were taken from
GenBank database, which contains more than ten thousands mRNA sequences.



\subsection{Profile analyses}
\subsubsection{Nucleotide distributions around start codons}
\vspace{2ex}
\noindent
\begin{tabular}{|l|}
\hline
Data analysis:\\
\hline
\end{tabular}

We have analyzed
 contextual nucleotide distributions of starting AUGs for all of vertebrates,
invertebrates and bacteria in the GenBank Database.
We also analyzed profile around start codons of mitochondria of
vertebrates and invertebrates. Because mitochondrial mRNA data were
less, we used 
DNA data. And to get enough data of mitochondrial sequences, we did
not exclude sequences with  
ATG trinucleotide in the same frame upstream.

\vspace{2ex}
\noindent
\begin{tabular}{|l|}
\hline
Result and discussion:\\
\hline
\end{tabular}

\begin{table}
\epsfile{file=gbvrta.ps,scale=0.80}
\caption{Profile around start codons in vertebrate}
\label{gbvrta}
\end{table}

\begin{table}
\epsfile{file=gbinvrta.ps,scale=0.80}
\caption{Profile around start codons in invertebrate}
\label{gbinvrta}
\end{table}

\begin{table}
\epsfile{file=bun_bct.ps,scale=0.80}
\caption{Profile around start codons in bacteria}
\label{bun_bct}
\end{table}


The results for vertebrates(table \ref{gbvrta})
 show gccGCC$^{\rm A}_{\rm G}$CC[AUG]G as the consensus
sequence and it supports Kozak's results\cite{label3}.
Especially purine in position -3 and guanine in position +4 are frequent
and they presumably play an important role in initiating translation.
The consensus sequence for invertebrates is AAA[AUG](table
\ref{gbinvrta}).  Position -3 in
invertebrates seems to be important as in vertebrates.

There are many Cs and Gs in positions more than 10 nucleotides upstream
in vertebrates. It may be due to translation regulation by forming 
secondary structure. But in invertebrate, this tendency is not observed.
Instead, there are many As and Ts in the same region. This will be
discussed later.

The results for bacteria(table \ref{bun_bct}) show that there is purine rich
region from  
position -14 to position -7, which may be due to the Shine-Dalgarno 
sequence. Unlike eucaryotes, there is no very strong consensus around
start codons. However, there are many A's in position -3 and position 
+4, which is consistent with previous research.


\begin{table}
\epsfile{file=vrta_mit.ps,scale=0.80}\\ \\
\epsfile{file=inv_mit.ps,scale=0.80}
\caption{Profile around start codons of mitochondria DNA}
\label{mit_prof}
\end{table}

Results for mitochondria in \ref{mit_prof} show that there are
consensus in several 
nucleotides upstream and much farther downstream. In mitochondria, only 15\%
of the genome is guanine according to our analysis. Still we suggest that 
guanine around start codons are especially less. The consensus
sequence is different from nucleus mRNA of eucaryotes in general.

\subsubsection{Nucleotide distributions around skipped AUGs}

\vspace{2ex}
\noindent
\begin{tabular}{|l|}
\hline
Data analysis:\\
\hline
\end{tabular}

We also analyzed nucleotide distributions around skipped AUGs to find
whether there is any remarkable consensus to promote leaky scanning.

\vspace{2ex}
\noindent
\begin{tabular}{|l|}
\hline
Result and discussion:\\
\hline
\end{tabular}

\begin{table}
\epsfile{file=lk_vrta.ps,scale=0.7}
\caption{Profile around skipped AUGs in vertebrates}
\label{lk_vrta}
\end{table}
\begin{table}
\epsfile{file=lk_invrta.ps,scale=0.7}
\caption{Profile around skipped AUGs in invertebrates}
\label{lk_invrta}
\end{table}

The tendencies around skipped AUGs (table \ref{lk_vrta} and
\ref{lk_invrta}) are different from ones
observed around start codons. We could not find any remarkable
patterns around skipped AUGs. This suggests that there is no consensus 
sequence to promote leaky scanning.

\subsection{Analyses of entropy value around start codons}

\vspace{2ex}
\noindent
\begin{tabular}{|l|}
\hline
Data analysis:\\
\hline
\end{tabular}

Entropy is known as a value which indicates the randomness.
If this value in a specific position in the sequence is low, then it 
means that the position has strong consensus sequence.
We calculated the entropy at each position(Position where A-residue of 
start AUG is located is defined 0.) for eucaryotes
and procaryotes (bacteria).
Bacteria data were further classified into {\it E.coli, Mycoplasma genitalium,
Haemophilus influenzae, and Bacillus subtilis}.  DNA sequence data have
been used for procaryotes and mRNA data for eucaryotes.  
Profile of nucleotide distribution around the start codon 
was constructed for each taxonomical group and procaryote species.
Each profile was then
normalized by total nucleotide frequencies.

The entropy at each position is given as follows:

\vspace{2ex}
\noindent
Let \( E_{i} \) be  the expected frequency of each
nucleotide \(i\) at the position, and
 \(O_{i} \) be the  observed frequency of each 
nucleotide in the profile. 
Let \( N_{i}\) be \(O_{i} / E_{i} \) .
Entropy at each position can then be defined as in the following equation.
\begin{quotation}
\noindent
\( \sum_{i = a,t,c,g} \frac{N_{i}}{N_{a} + N_{t} 
+ N_{c} + N_{g}} \log{ \frac{N_{i}} {N_{a} + N_{t} + N_{c} + N_{g}} }
\)  \\
\end{quotation}

The normalization was necessary,
because some species  have
a characteristically low/high G+C content, which would make the whole
entropy values of the species lower.

\vspace{2ex}
\noindent
\begin{tabular}{|l|}
\hline
 Result and discussion:\\
\hline
\end{tabular}

\begin{figure}
\epsfile{file=euc_ent.ps,hscale=0.7,vscale=0.5}\\
\epsfile{file=bct4ent2.ps,scale=0.7}
\caption{Normalized entropy around start codons}
\label{entro}
\end{figure}

The results are shown in figure \ref{entro}.
For all the eucaryotes, the entropy values are very low at the
positions -3 (3 base upstream from the start codon), which is
consistent with the Kozak rule\cite{label3}.  The position +3 (one
base downstream from the start codon, ATG) has modestly low entropy
value, and for the position -4 and its upstream, the entropy values
are relatively high, with small fluctuation.

For procaryotes, the
entropy value is considerably low at the position -11 in {\it bacillus
subtilis} and position -9 in {\it E. coli} and {\it H. influenzae},
respectively.  The sequences around the positions are presumably
Shine-Dalgarno sequences.  The curve representing {\it Mycoplasma
genitalium}, however, does not decrease at these positions.

\subsection{GC and AT content in 5'UTR}

Results for profile analyses showed that there is a GC-rich region in
5'UTR in vertebrates, and AT-rich region in invertebrates, which are
consistent with previous researches. The GC-richness in vertebrates
is presumably due to translation regulation, and the GC-richness
partly depends on what kind of proteins will be synthesized. For
example, mRNA sequences that have GC-richness over 70\% will often
produce growth factor, proto-oncogenes, receptor proteins,
housekeeping genes,etc\cite{label12}. Approximately 20\% of vertebrate
mRNA sequences have GC-richness over 70\%. But the exact reason for
AT-richness in invertebrate mRNA sequences is unknown.
Here, we first analyze distribution of GC content in vertebrates by
using sequences from GenBank to make sure that the result is
consistent with previous research and then we analyze distribution of
AT content in invertebrates to compare with that of vertebrates.

\vspace{2ex}
\noindent
\begin{tabular}{|l|}
\hline
 Data analysis:\\
\hline
\end{tabular}

We used mRNA sequences whose 5'UTRs are longer than 30 bases for the 
precise analysis. Then we made histograms that indicate the rate of
mRNA sequences that have specific amount of GC/AT content.

\vspace{2ex}
\noindent
\begin{tabular}{|l|}
\hline
 Result and discussion:\\
\hline
\end{tabular}

\begin{figure}
\begin{center}
\epsfile{file=gc_vrta_5utr.ps,scale=0.5}\\
\epsfile{file=at_invrta_5utr.ps,scale=0.5}
\end{center}
\caption{Rate of specific amount of GC/AT content in 5'UTR}
\label{gc_at} 
\end{figure}

Although we analyzed GC content for vertebrates, and AT content for
invertebrates, we discovered that the shape of the curve representing
the distribution 
of GC content in vertebrates is similar to the shape of the curve
representing the distribution of AT content in invertebrates(figure
\ref{gc_at}). 
The reason is unknown, but AT richness in invertebrate 5'UTRs may have
some roles as much as the roles of GC richness in vertebrates.
The inevitable consequence is that such 5'UTR sequences in
invertebrates lack extensive
secondary structure. And according to previous researches, it seems
that in lower eucaryotes, secondary 
structure disturbs translation much more than in higher eucaryotes.
Thus, AT richness in invertebrates may be the positive control of
regulation of protein synthesis at translation level, whereas the
GC richness in vertebrates are negative control. 
Figure \ref{at_40} and \ref{at_80} show the examples of proteins
that are synthesized
from invertebrate mRNA whose AT content is below 40\% and above 80\%.  
Further investigation on the common characteristics of these proteins
must be done.

\begin{table}
\begin{tiny}
\begin{center}
\begin{tabular}{|l|}
\hline
 AT content: 22.86 \% : silk gland factor-1 (SGF-1)
\\ AT content: 23.53 \% : apolipophorin-III
\\ AT content: 23.89 \% : surface protease
\\ AT content: 24.22 \% : surface protease
\\ AT content: 24.43 \% : surface protease
\\ AT content: 27.24 \% : homeotic protein
\\ AT content: 27.50 \% : major surface glycoprotein
\\ AT content: 28.81 \% : major surface glycoprotein
\\ AT content: 30.66 \% : Lazarillo precursor
\\ AT content: 33.33 \% : synaptotagmin, p65
\\ AT content: 34.38 \% : red pigment-concentrating hormone
\\ AT content: 34.62 \% : surface glycoprotein
\\ AT content: 35.19 \% : blackjack
\\ AT content: 35.76 \% : major surface glycoprotein
\\ AT content: 36.11 \% : cocaine-sensitive serotonin transporter
\\ AT content: 37.04 \% : Shaw potassium channel
\\ AT content: 37.18 \% : surface glycoprotein
\\ AT content: 37.18 \% : surface glycoprotein
\\ AT content: 37.32 \% : major surface glycoprotein
\\ AT content: 37.65 \% : insulin-like peptide
\\ AT content: 37.80 \% : profilin II
\\ AT content: 37.84 \% : profilin I
\\ AT content: 38.27 \% : ADP/ATP carrier protein
\\ AT content: 38.46 \% : molt-inhibiting hormone precursor
\\ AT content: 38.46 \% : rac1 protein
\\ AT content: 38.46 \% : rac1 protein
\\ AT content: 38.71 \% : hemolysin
\\ AT content: 39.32 \% : POU domain protein
\\
\hline
\end{tabular}
\end{center}
\end{tiny}
\caption{Invertebrate mRNAs that have AT content lower than 40\% in
5'UTR, and proteins that will be synthesized}
\label{at_40}
\end{table}

\begin{table}
\begin{tiny}
\begin{center}
\begin{tabular}{|l|}
\hline
 AT content: 80.22 \% : telomerase component p80
\\ AT content: 80.33 \% : Der p V allergen
\\ AT content: 80.70 \% : peripheral membrane protein
\\ AT content: 81.25 \% : BMP receptor
\\ AT content: 81.48 \% : GDP dissociation inhibitor
\\ AT content: 81.61 \% : Sarcophaga pro-cathepsin B
\\ AT content: 81.90 \% : tropomyosin
\\ AT content: 82.50 \% : myosin heavy chain
\\ AT content: 82.98 \% : glucose-6-phosphate dehydrogenase
\\ AT content: 83.58 \% : sapecin B
\\ AT content: 83.72 \% : cytochrome c oxidase III
\\ AT content: 84.00 \% : S-antigen
\\ AT content: 84.09 \% : prophenoloxidase subunit 2
\\ AT content: 84.31 \% : apocytochrome b
\\ AT content: 84.38 \% : hyphancin IIID
\\ AT content: 84.54 \% : beta-tubulin
\\ AT content: 84.75 \% : glucose-6-phosphate dehydrogenase
\\ AT content: 84.78 \% : paramyosin
\\ AT content: 84.78 \% : telomerase component p95
\\ AT content: 85.11 \% : hyphancin IIIG
\\ AT content: 85.31 \% : S-adenosylhomocysteine hydrolase
\\ AT content: 85.37 \% : fimbrin
\\ AT content: 85.61 \% : vacuolar ATPase subunit DVA41
\\ AT content: 86.02 \% : glycoprotein FP21
\\ AT content: 86.23 \% : histone H1
\\ AT content: 86.67 \% : opsin
\\ AT content: 86.92 \% : actin
\\ AT content: 86.99 \% : calcium binding protein
\\ AT content: 87.14 \% : protein antigen
\\ AT content: 87.45 \% : glycoprotein 185
\\ AT content: 87.50 \% : hyphancin IIIE
\\ AT content: 87.54 \% : ornithine aminotransferase
\\ AT content: 87.73 \% : integral membrane protein
\\ AT content: 88.35 \% : serine repeat protein
\\ AT content: 88.89 \% : histone H3
\\ AT content: 88.89 \% : triosephosphatee isomerase
\\ AT content: 89.06 \% : NADH dehydrogenase subunit 8
\\ AT content: 89.17 \% : major merozoite surface antigen
\\ AT content: 89.74 \% : hyphancin IIIF
\\ AT content: 90.32 \% : integral membrane protein
\\ AT content: 90.43 \% : ribonucleotide reductase small subunit
\\ AT content: 90.48 \% : cyclophilin
\\ AT content: 90.54 \% : succinyl coenzyme A synthetase alpha subunit
\\ AT content: 90.79 \% : circomsporozoite-related antigen
\\ AT content: 90.91 \% : H(+)-transporting ATPase
\\ AT content: 92.31 \% : WD40 repeat protein 2
\\ AT content: 93.10 \% : calcineurin
\\ AT content: 94.29 \% : cecropin A precursor
\\ AT content: 95.24 \% : cyclase associated protein
\\
\hline
\end{tabular}
\end{center}
\end{tiny}
\caption{Invertebrate mRNAs that have AT content higher than 80\% in
5'UTR, and proteins that will be synthesized}
\label{at_80}
\end{table}


\section{On Low Frequencies of AUG Trinucleotides in front of Start Codons}
\label{lowfreq}

In this section, we focus on the skipped AUGs that are close to start
codons.
Ribosomes ignore AUGs that should not serve as start codons by leaky 
scanning and they select appropriate AUGs as start codons. But do
ribosomes confuse to select appropriate AUG as start codon if two AUGs
are close to each other? 
% Some experiments showed that AUG
% trinucleotide just in front of start codon will disturb translation
% initiation from the appropriate start codon in some organisms such as 
% yeast $Saccharomyces cerevisiae$\cite{label30}.
 We investigated this hypothesis from
4 approaches. That is frequencies of AUG trinucleotides around start codons,
rate of leaky scanning when two AUGs are close to each other,
consensus sequences when two AUGs are close, and relationship between
translation initiation mechanisms and evolutions.


\subsection{Frequencies of AUG trinucleotides around start codons}
 

Frequencies of AUG trinucleotides around start codons are calculated
to find whether the frequencies decrease around start codons.

\vspace{2ex}
\noindent
\begin{tabular}{|l|}
\hline
Hypothesis:\\
\hline
\end{tabular}

Ribosomes confuse to select appropriate AUG as 
start codon if two AUGs are close to each other. Thus, the frequencies
of AUG trinucleotides around start codons are relatively low to
prevent the wrong selection of the AUGs.

\vspace{2ex}
\noindent
\begin{tabular}{|l|}
\hline
Data analysis:\\
\hline
\end{tabular}

Computer analysis was done to analyze the
frequencies of AUG trinucleotides around start codons. Except for
bacteria, mRNA data was used. For bacteria, DNA data was used, because
mRNA data for bacteria is very few. The frequency of position \(F_{p}\) is
calculated as follows.

\begin{displaymath}
F_{p} = \frac{Number\:of\:sequences\:that\:has\:AUG\:in\:position\:p}
      {Number\:of\:sequence\:data\:that\:has\:data\:for\:position\:p}
\end{displaymath}

And then smoothing was done for every 3 positions to reduce too much 
fluctuations. Because sequences with AUG trinucleotides located
upstream of start 
codons and in the same reading frame(every 3 positions from start
codon)are excluded for the precise analyses(described in
\ref{method_3}), we multiplied the smoothed  
value by \(\frac{3}{2}\) for upstream.

If the hypothesis holds, then the frequency of AUG trinucleotide gets
lower as the position in mRNA gets closer to start codon.
If not, the frequencies of AUG trinucleotides upstream will be constant for
upstream, and the frequencies of AUG trinucleotides downstream will
also be constant.

\vspace{2ex}
\noindent
\begin{tabular}{|l|}
\hline
Results:\\
\hline
\end{tabular}

\begin{figure}
\epsfile{file=pri_atg.ps,scale=0.40}
\epsfile{file=rod_atg.ps,scale=0.40}\\
\epsfile{file=mam_atg.ps,scale=0.40}
\epsfile{file=vrt_atg.ps,scale=0.40}\\
\epsfile{file=inv_atg.ps,scale=0.40}
\epsfile{file=bct_atg.ps,scale=0.40}
\caption{Frequencies of AUG trinucleotides around start codons}
\label{atg_freq}
\end{figure}

The results(figure \ref{atg_freq}) show that there is a general
tendency to have fewer  
AUG trinucleotides in front of start codons for all taxonomical
groups.
In eucaryotes, the frequencies of AUGs just downstream of start codon is low,
comparing to the frequencies far downstream, but the frequencies are
much higher comparing to ones in front of start codons.
In other words, the frequency drastically increases at the position
 of start codon for
all eucaryotes. But in bacteria, the frequencies of AUG trinucleotides
in positions just several nucleotides downstream of start codon is still low.

\vspace{2ex}
\noindent
\begin{tabular}{|l|}
\hline
Discussion:\\
\hline
\end{tabular}

The general tendency of the low frequency near start
codon in eucaryotes is consistent
with the hypothesis. The reason that the frequency drastically
increases at the position of start codon for all eucaryotes is
presumably
because according to the scanning model in eucaryotes,
ribosome scans mRNA from 5' end to downstream, searching for AUG.
If the AUG trinucleotide is located upstream of start codon, 
leaky scanning must work to prevent the initiation from this
inappropriate AUG. But if this mechanism is not reliable, 
those AUG trinucleotides should not exist upstream of start codons.  
But once translation starts
from start codon, AUG trinucleotide downstream of start codon has no 
influence on translation initiation.
Thus, this may be the reason that the frequencies of AUG
trinucleotides upstream of start codon is low and the frequencies
 of AUG trinucleotides downstream of start codon
is relatively high comparing to ones in front of start codons.

In bacteria, however, it is thought that ribosome binds directly to 
the translation initiation site instead of scanning mRNA from upstream
to downstream. Thus, we suggest that in bacteria, 
ribosome is likely to mistake nearby downstream AUGs for start codon
as well as nearby upstream AUGs.


\subsection{Rate of leaky scanning and distance between two AUGs}


We have investigated the relationship between distances of two AUGs
and the rate of leaky scanning.

\vspace{2ex}
\noindent
\begin{tabular}{|l|}
\hline
Hypothesis:\\
\hline
\end{tabular}

When two AUGs are close to each other, ribosome confuse to select
one of AUGs as start codon. This phenomenon has influence on leaky
scanning. 
Thus rate of leaky scanning changes when two AUGs are close to each other.

\vspace{2ex}
\noindent
\begin{tabular}{|l|}
\hline
Data analysis:\\
\hline
\end{tabular}

The rate of leaky scanning when two AUGs are separated by \(n\) bases
(\(RLK_{n}\)) is calculated as follows.

\begin{small}
\begin{displaymath}
RLK_{n} = \frac{Number\:of\:leaky\:scanning\:observed\:when\:two\:AUGs\:
are\:separated\:by\:n\:bases}{Number\:of
\:cases\:that\:there\:is\:AUG\:trinucleotide\:n\:bases\:
upstream/downstream\:of\:start\:codon}
\end{displaymath}
\end{small}

\vspace{2ex}
\noindent
\begin{tabular}{|l|}
\hline
Results:\\
\hline
\end{tabular}

\begin{figure}
\begin{center}
\epsfile{file=leak_bun.ps,scale=0.40}
\end{center}
\caption{Relationship between leaky scanning and distance between two AUGs}
\label{leak_bun}
\end{figure}

The rate of leaky scanning fluctuates(figure \ref{leak_bun}), but this holds
not only when 
two AUGs are close to each other, but also when two AUGs are separated
far from each other.

While the first AUG is selected as a
starting AUG for the most of the time, we found that the second AUG is
frequently selected by leaky scanning if the two AUG's are
separated by \(3n+2\) bases where \(n\) is an integer.  

\vspace{2ex}
\noindent
\begin{tabular}{|l|}
\hline
Discussion:\\
\hline
\end{tabular}

The distance of two AUGs does have influence on rate of leaky
scanning, but we cannot observe any remarkable features when two AUGs
are close to each other. Thus, the hypothesis is not confirmed.

The periodic peak of the rate is presumably due to the following
hypothesis(figure \ref{3n2}).
Suppose that AUGs are separated by \(3n+2\) bases and translation
starts from first AUGs. The nucleotide pattern ``ua'' and ``ug''
just before second AUG trinucleotide and ``a'' just after the second 
AUG will make stop codons in frame, disturbing the farther translation
downstream. Thus, translation initiation from first AUG when second
AUG is separated by \(3n+2\) bases restricts these nucleotide patterns
around second AUGs. This restriction may make the rate of leaky scanning 
higher.

\begin{figure}
\begin{picture}(400,120)
\put(0,73){\line(1,0){50}}
\put(50,70){AUG}
\put(73,73){\line(1,0){200}}
\put(273,70){\underline{taA}UG}
\put(305,73){\line(1,0){50}}


\put(0,43){\line(1,0){50}}
\put(50,40){AUG}
\put(73,43){\line(1,0){200}}
\put(273,40){\underline{tgA}UG}
\put(305,43){\line(1,0){50}}



\put(0,13){\line(1,0){50}}
\put(50,10){AUG}
\put(73,13){\line(1,0){210}}
\put(282,10){A\underline{UGa}}
\put(311,13){\line(1,0){44}}

\put(150,80){$3n+2$ bases}
\put(240,81){\shortstack[c]{\small Stop codons in\\ the reading frames}}
\end{picture}

\caption{When two AUGs are separated by \(3n+2\) base pairs, stop
codons are likely to be in the same reading frame.}
\label{3n2}
\end{figure}








\subsection{Tendencies when two AUGs are close to each other}

The frequencies of AUG trinucleotides around start codons
are low from our results. Thus, existing AUGs around start codons 
may have some special features to prevent or promote leaky scanning in 
order to initiate translations from the appropriate AUG. So the
following hypothesis was made.

\vspace{2ex}
\noindent
\begin{tabular}{|l|}
\hline
Hypothesis:\\
\hline
\end{tabular}

 In case where two AUGs are close to each other, strong
consensus sequence exists to prevent or to promote leaky scanning.

\vspace{2ex}
\noindent
\begin{tabular}{|l|}
\hline
Data analysis:\\
\hline
\end{tabular}

Calculation of the entropy of positions which is 3 bases upstream from 
AUG trinucleotides were done for the cases where
two AUGs are separated by specific number of bases.
Note that position -3 is thought to be the important position for the
translation initiation.
This calculation was done for the following 4 kinds of positions.\\ 
\\

\noindent
{\large Normal Scanning}\\
\begin{picture}(400,50)
\put(0,23){\line(1,0){50}}
\put(50,20){AUG}
\put(73,23){\line(1,0){60}}
\put(132,20){AUG}
\put(155,23){\line(1,0){244}}

\put(50,33){\vector(1,0){300}}
\put(50,38){Translation}

\put(40,10){\vector(0,1){8}}
\put(122,10){\vector(0,1){8}}

\put(20,2){Position A}
\put(102,2){Position B}

\end{picture}\\ \\

\noindent
{\large Leaky Scanning}\\
\begin{picture}(400,50)
\put(0,23){\line(1,0){50}}
\put(50,20){AUG}
\put(73,23){\line(1,0){60}}
\put(132,20){AUG}
\put(155,23){\line(1,0){244}}

\put(132,33){\vector(1,0){217}}
\put(132,38){Translation}

\put(40,10){\vector(0,1){8}}
\put(122,10){\vector(0,1){8}}

\put(20,2){Position C}
\put(102,2){Position D}

\end{picture}\\ \\


\begin{enumerate}
\item Position A: Position  -3 in normal scanning.
\item Position B: Position which is 3 bases upstream from AUG trinucleotide
located downstream of start codon.
\item Position C: Position which is 3 bases upstream from AUG
trinucleotide
located upstream of start codon.
\item Position D: Position -3 in leaky scanning.
\end{enumerate}

Note that if strong consensus is observed in position A or D, that may be
the strong signal to initiate translation from that position.
If strong consensus is observed in postion B or C, that may be the
strong signal to prevent initiation from that position.

Because data for
invertebrate are less, we analyzed only vertebrates.

\vspace{2ex}
\noindent
\begin{tabular}{|l|}
\hline
Results:\\
\hline
\end{tabular}

\begin{figure}
\begin{center}
\epsfile{file=m3entabcd.ps,scale=0.40}
\end{center} 
\caption{Entropies of the positions that are 3 bases upstream from AUG 
trinucleotides}
\label{m3ent}
\begin{picture}(400,0)
\put(150,53){\tiny 2}
\end{picture}
\end{figure}

Figure \ref{m3ent} shows entropy of 4 kinds of positions in vertebrates.
The entropy of position -3 in general is 1.47 and
the entropy of the position which is 3 bases upstream
from the skipped AUGs in general is 2.0(not normalized according to GC
content).  
As there were not much data, entropy value tend to be low. But
entropies of position B and position C tend to be high comparing with
position A and position D.
And from this figure, we can observe that when two AUGs are separated
 by two bases, entropy of position -3 is very low(See position A in
figure \ref{m3ent}).

\begin{table}
\begin{center}
\epsfile{file=atg_2_atg.ps,scale=0.70}
\end{center}
\caption{Nucleotide distributions around two AUGs that are exactly
2 base pair apart.Normal scanning (above) and leaky scanning (below).}
\label{atg_2_atg}
\end{table}

Table \ref{atg_2_atg} shows nucleotide distribution around two AUG's that are
two bases apart in vertebrates.  
The first table is for the cases where the
first AUG is selected, and the second table is
for the cases where the second AUG is selected, skipping the first.
In case where starting AUG is followed by second AUG with 2 bases in between, 
ACC[AUG]G is observed as 
a very "strong" consensus sequence around the starting AUG. Furthermore,
there are no thymine or cytosine in position -3. If the tendency of this
position is same as that of the vertebrate mRNAs in general, 
the possibility that this
occurs is less than 1\%. 


\vspace{2ex}
\noindent
\begin{tabular}{|l|}
\hline
Discussion:\\
\hline
\end{tabular}

As the entropies of position B and C is generally high, we suggest
that there is no strong consensus sequence at these positions to
prevent translation from non-starting AUG even when two AUGs are close 
to each other. But according to entropy of position A and position D,
there is consensus sequence at these positions to promote translation
from start codons. And we suggest that position -3 is important to
avoid leaky scanning, especially when two AUGs are separated by two bases.










\subsection{Relationship between initiation mechanism and evolution}

In this subsection, investigation about whether there is any
relationship between evolution and translation initiation mechanism
will be conducted.

\vspace{2ex}
\noindent
\begin{tabular}{|l|}
\hline
Hypothesis:\\
\hline
\end{tabular}

There is a relationship between the evolution and distance between
start codon and upstream AUG trinucleotide.

\vspace{2ex}
\noindent
\begin{tabular}{|l|}
\hline
Data analysis:\\
\hline
\end{tabular}

We have calculated the average distances between a start codon and the nearest
non-starting ATG trinucleotide to the upstream and to the
downstream. This time, we have used DNA data rather than mRNA data in
GenBank because a large amount of data are required for precise
analyses and there are more sequences as DNA data than as mRNA data.
For the sake of comparison, distances between  start codons and other
trinucleotides consisting of 'A','T','G' are measured.

\vspace{2ex}
\noindent
\begin{tabular}{|l|}
\hline
Results:\\
\hline
\end{tabular}

\begin{figure}
\epsfile{file=atg_dod2.ps,scale=0.8}
\caption{Average distances between start codons and ATG trinucleotides 
upstream/downstream}
\label{atg_dod2}
\end{figure}

\begin{figure}
\epsfile{file=atg_dist.ps,scale=0.7}
\caption{Average distances between start codons and specific
trinucleotides upstream}
\label{atg_dist}
\end{figure}

\begin{figure}
\epsfile{file=atg_disd.ps,scale=0.7}
\caption{Average distances between start codons and specific
trinucleotides downstream}
\label{atg_disd}
\end{figure}

The results show that for the upstream, the average distance
for AUG trinucleotide is the longest of all the other 
five trinucleotide patterns in eucaryotes(figure \ref{atg_dist}).
Furthermore, the average distances 
are generally longer in higher organisms
than in lower organisms; more specifically in the following order:
primates $>$ rodent $>$ mammals $>$ vertebrates $>$ invertebrates $>$
bacteria.  This rule  holds only  for AUG trinucleotides upstream,
not downstream(figure \ref{atg_dod2}).


\vspace{2ex}
\noindent
\begin{tabular}{|l|}
\hline
Discussion:\\
\hline
\end{tabular}

The result shows that there is a relationship between evolutions
and average distance upstream. 
One possible hypothesis we can
think for the reason 
that the distances of higher organisms are longer is as follows:
In bacteria, ribosomes bind directly to translation initiation site.
But in eucaryotes, ribosomes must first bind to CAP and look for
translation initiation site. Furthermore, bacteria mRNAs are
polycistronic and the coding regions may even overlap. And ribosomes
recognize translation initiation site accurately. From this point of
view, lower organisms 
may have more sophisticated and more reliable mechanism for translation 
initiation site. Thus ribosomes of lower organisms are unlikely to
confuse. 
We suggest that this is why distance between start codon and upstream 
AUG can be shorter in lower organisms.













\subsection{Discussion of the cases where two AUGs are close}

From our results, we suggest that there is evolutional pressure to 
eliminate AUG trinucleotides near start codons, presumably because AUG 
trinucleotides near start codons may confuse ribosomes. This
hypothesis is consistent with recent experimental research on yeast
$Saccharomyces\:cerevisiae$\cite{label30}.
The tendency to keep the AUG trinucleotides far from start codons
is
weaker in lower organisms, presumably because lower organisms have
more reliable mechanisms to identify the appropriate AUG as a start codon.

However the existing AUGs, which are close to each other, does not
seem to have much influence on the fidelity of translation initiation. 
%Thus the
%hypothesis that upstream AUGs near start codons are
%unpreferable holds in the long period of time. 
But in cases where two
AUGs are separated by two bases, strong consensus sequences may be 
needed to avoid leaky scannings.



\section{Frequencies of Stop Codons Located Downstream of Skipped AUGs}
\label{reinitia}
Although cytosine or thymine in position -3 is known to promote leaky
scanning, our results showed that there is no remarkable consensus sequence to
promote leaky scanning. Then the question ``what promotes leaky scanning 
for the existing AUGs located upstream of start codons?'' arises.
It is known that leaky scanning occurs if
the first AUG is followed shortly by a stop codon(In other words
reinitiation occurs). In this section, we
focus on stop codons which is located downstream of the skipped
AUGs. 

\vspace{2ex}
\noindent
\begin{tabular}{|l|}
\hline
Hypothesis:\\
\hline
\end{tabular}

A lot of existing AUGs in 5'UTR are skipped by ribosomes by the stop
codons located just after in the same frame. Thus the frequency of
stop codons after the skipped AUGs in the same frame will be
relatively high. 


\vspace{2ex}
\noindent
\begin{tabular}{|l|}
\hline
Data analysis:\\
\hline
\end{tabular}

Frequencies of stop codons located downstream of first skipped AUGs
in position \(p\)(\(FS_{p}\)) are calculated by the following
formula(Position of A located in skipped AUG is defined +1.). 


\begin{displaymath}
FS_{p} =
\frac{Number\:of\:stop\:codons\:that\:appear\:in\:position\:p}{
Number\:of\:sequence\:data\:that\:has\:data\:for\:position\:p}
\end{displaymath}

We also calculated the rate of mRNAs that have stop codons located
downstream of the first skipped AUGs in the same frame and located
within specific distance \(d\)(\(RR_{d}\)) to investigate how many
skipped AUGs that can be explained by reinitiation mechanisms(Distance
\(d\) is defined as position \(d\)).

\begin{displaymath}
RR_{d} =
\frac{mRNAs\:whose\:first\:skipped\:AUGs\:have\:stop\:codons\:within\:distance\:d}{mRNAs\:that\:have\:skipped\:AUGs} 
\end{displaymath}

If a second AUG trinucleotide located
upstream of start codon is flanked by first AUG trinucleotide and stop 
codon in the same frame(in other words, if the second AUG
trinucleotide is located in the ``minicistron'' upstream of start
codon),  this AUG trinucleotide
may not have any effect on 
translation initiation. Thus, for the simplicity of the calculation, we
only focused on first skipped AUGs.
And we only calculated the cases where stop codons located downstream
of skipped AUG are located upstream of start codons, because the
influences to reinitiation by stop codons 
that are located downstream of start codons are unknown.


\vspace{2ex}
\noindent
\begin{tabular}{|l|}
\hline
Results:\\
\hline
\end{tabular}

\begin{figure}
\begin{center}
\epsfile{file=eu_stopf.ps,scale=0.40}
\end{center}
\caption{Frequencies of stop codons located downstream of the first
skipped AUGs and in the same frame as the first skipped AUGs}
\label{eu_stopf}
\end{figure}


\begin{figure}
\epsfile{file=gbstop.ps,scale=0.67}
\caption{Frequencies of stop codons located downstream of the first
skipped AUGs in vertebrates}
\label{gbstop}
\end{figure}

\begin{figure}
\epsfile{file=invstop.ps,scale=0.67}
\caption{Frequencies of stop codons located downstream of the first
skipped AUGs in invertebrates}
\label{invstop}
\end{figure}

\begin{figure}
\begin{center}
\epsfile{file=stop_reini.ps,scale=0.40}
\end{center}
\caption{Rate of mRNAs that have stop codons located downstream of
the first skipped AUGs in the same frame within specific distance}
\label{stop_reini}
\end{figure}

Figure \ref{eu_stopf} shows frequency of stop 
codons after the first skipped
AUGs in the same reading frame.  The
stop codon frequencies gradually decrease in
both vertebrates and invertebrates.  Also according to the figure
\ref{gbstop} and \ref{invstop}, there are regular peaks of
the frequency every 3 base positions(corresponding bar is colored in
 black in the figure)
 from the skipped AUG.  In
other words, stop codons exist preferably in the same reading frame
as the skipped AUG. 

Figure \ref{stop_reini} shows rate of mRNAs that have stop codons
located downstream of the first skipped AUGs in the same frame and within
specific distance from the first skipped AUGs. 



\vspace{2ex}
\noindent
\begin{tabular}{|l|}
\hline
Discussion:\\
\hline
\end{tabular}

From these results, stop codons located downstream 
and close to AUG trinucleotides may be important for the leaky scannings.
From figure \ref{stop_reini}, if
the ribosomes have ability to reinitiate translation when stop codons
are located within position +20 from AUG trinucleotides, 
\(\frac{1}{4}\) of the cause of leaky scanning may be explained by
reinitiation mechanism. Although it is believed that this distance
must be short, we cannot suggest the appropriate
distance. In human immunodeficiency virus type 1 mRNA, if the upstream 
open reading frame consists of 84 nucleotides, reinitiation occurs by
50\%\cite{label31}.  
% In Kozak's paper, it is written that reinitiation occurs
% when AUG trinucleotide is followed shortly by stop codons.






\newpage
% \setcounter{page}{30}
\chapter{Conclusion}
\label{concl}


Our question was how the organisms determine translation initiation
sites. We suggested that nucleotide patterns in translation initiation 
site play
important role for the translation initiation.
To investigate it, series of computational experiments have
been conducted to analyze tendencies of nucleotide distributions
around start codons and skipped AUGs for various species and
taxonomical groups. And we have found that there are some remarkable
patterns around start codons for the most of organisms, which is
consistent with the previous research. But we did not find remarkable
patterns around skipped AUGs, which may promote leaky scannings.

Next, we made a hypothesis that in eucaryotes, if two AUGs are located close
to each other, 
ribosomes will confuse to select the appropriate AUG as the start
codons. We investigated it from four approaches. Then the result
showed that there is a tendency to have fewer AUG trinucleotides in
front of start codons. This indicates that AUG trinucleotides just in 
front of start codons are unpreferable for the appropriate selection
of start codons. 
However existing AUG trinucleotides just in front of
start codons do not seem to have much influence on fidelity of the selection of
the appropriate AUGs. 
And the interesting discovery is that higher organisms tend to place
AUG trinucleotides farther from start codons. One interpretation is
that this is because higher organisms do not have much sophisticated
mechanism for the translation initiation. This hypothesis may be
extended to mechanisms of organisms at molecular level in
general. Eucaryotes have much DNAs 
that do not code protein. In other words, genome of higher
organisms are redundant, if non-coding region of DNA does not have
much role. As discussed in information theory, redundancies will
permit errors\footnote{According to the information theory, if the
minimum Hamming distance is more than 
\(2t + 1\), \(t\) errors can be corrected.}. On the other
hand, genomes of procaryotes are less 
redundant. Thus we suggest that for procaryotes, less errors are
permissible and mechanisms must be much sophisticated to prevent errors.

Finally, we have analyzed the distributions of stop codons located
downstream of skipped AUGs and suggested that those stop codons are
presumably playing some important role for leaky scannings.

In this paper, we discovered tendencies that may be involved in
translation initiation. But further analyses from various view points
must be done to understand about the mechanism of translation
initiation. Computer scientific algorithms may be useful to
discover some knowledge about translation initiation from the databases.
And if we can model the translation initiation mechanisms simply,
computer simulation may be helpful to discover some factors 
involved in translation initiation.


% \begin{itemize}
% \item Tendencies involved in translation initiations are discovered.
% \item Further analyses from various point of view must be done to 
% understand completely about translation initiation.
% \item Computer scientific algorithms must be applied to discover rules
% from the databases.
% \end{itemize}

\chapter*{Acknowledgements}
The author thanks to associate professor Masaru Tomita for giving me
suggestions and comments. And thanks to professor Takemochi Ishii and
associate professor Yoshiyasu Takefuji for giving me a comment on this
paper. And thanks to Yoshimi Toda and Junko Hara for checking my
paper. And thanks to my project members, Hidekazu Sasaki, Yuko Osada,
and Yukari Shimizu for working with me in the project.

\newpage
\begin{thebibliography}{99}
% \addcontentsline{toc}{chapter}{References}
\bibitem{label0} Bruce Alberts et al.,
Molecular Biology of THE CELL third edition,
Garland Publishing Inc.

\bibitem{label3} Kozak,M.(1987) An analysis of 5'-noncoding sequences from
699 vertebrate messenger RNAs,Nucleic Acids Research Volume 15 pp.8125-8148

\bibitem{label1} Kozak,M.(1995),Adherence to the first-AUG rule when a 
second AUG codon follows closely upon the first,
Proc.Natl.Acad.Sci.USA Vol.92 pp.2662-2666

\bibitem{label2} Kozak,M.(1990),Downstream secondary structure facilitates
recognition of initiator codons by eukaryotic ribosomes,
Proc.Natl.Acad.Sci.USA VOl.87.pp.8301-8305

\bibitem{label4} Kozak,M.(1989),The Scanning Model for Translation:An Update,
J.Cell.Biol 108,pp.229-241

\bibitem{label6} Kozak,M.(1989),Context Effects and Inefficient Initiation
 at Non-AUG Codons in Eucaryotic Cell-Free Translation Systems,

\bibitem{label12} Kozak,M.(1992),Regulation of Translation in Eucaryotic
Systems,Annuu.Rev.Cell Biol.8:197-225

\bibitem{label5} Slunsher, L. et al(1991),mRNA leader length and initiation
codon context determine alternative AUG selection for the yeast gene MOD5,
Proc.Natl.Acad.Sci.USA Vol.88,pp.9789-9793


\bibitem{label8} Imataka,H. et al(1994), Cell-specific Translational Control
of Transcription Factor BTEB Expression,J.Biol.Chem. Vol.269 pp.20668-20673

\bibitem{label7} Shine,J. and Dalgarno,L.(1974). The 3'-terminal sequence
of Escherichia coli 16S ribosomal RNA:Complementarity to nonsense triplets
and ribosome binding sites. Proc.Nat.Acad.Sci.USA Vol.71 pp1342-1346

\bibitem{label9} Hongyun Chen, et al(1994),
Determination of the optimal aligned spacing between the Shine-Dalgarno
sequence and the translational initiation codon of Escherichia coli mRNAs,
Nuc.Acid.Res. Vol.22 pp.4953-4957

\bibitem{label10} Rinke-Appel,J. et al(1994),
 Contacts between 16S ribosomal RNA and
mRNA, within the spacer region separating the AUG initiator codon and the
Shine-Dalgarno sequence; a site directed cross-linking study,
Nuc.Acid.Res. Vol.22 pp.3018-3025

\bibitem{label11} Schneider,T et al.(1986),
Information Content of Binding Sites on Nucleotide Sequences,
J.Mol.Biol.188,pp.415-431

\bibitem{label18} Kozak,M.(1987),
 Effects of intercistronic length on the efficiency of reinitiation
by eucaryotic ribosomes,
Mol.Cell.Biol.10,pp.3438-3445

\bibitem{label19} Kozak,M.(1991),
 A short leader sequence impairs the fidelity of initiation by eukaryotic
ribosomes, Gene Expr, 2, pp.111-115

\bibitem{label20} Claudio O.Gualerzi and Cynthia L.Pon(1990),
Initiation of mRNA Translation in Procaryotes,
Biochem.Vol.29,pp.5881-5889

\bibitem{label21} Kozak,M.(1986),
Influences of mRNA secondary structure on initiation by eucaryotic 
ribosomes,
Proc.Natl.Acad.Sci.83,pp.2850-2854


\bibitem{label22} Kozak,M.(1983)
Comparison of initiation of protein synthesis in procaryotes, eucaryotes,
and organelles,
Microbiol Rev Vol.47 pp.1-45

\bibitem{label23} IG.Ivanov, et al.(1995)
A second putative mRNA binding site on the Escherichia coli ribosome.
Gene 160:75-79

\bibitem{label24} Hannig,EM.(1995)
Protein synthesis in eucaryotic organisms: new insights into the function
of translation initiation factor eIF3.
Bioessays 17:915-919

\bibitem{label25} Herman, A.DE BOER et al.(1990)
Sequences within Ribosome Binding Site Affecting Messenger RNA Translatability
and Methods to Direct Ribosomes to Single Messenger RNA Species.
Methods Enzymol. 185:103-114

\bibitem{label26} Maarten H. de Smit. et al.(1994)
Translational Initiation on Structured Messengers.
Another Role for the Shine-Dalgarno Interaction.
J.Mol.Biol. 235:173-184

\bibitem{label27} Claire M.Fraser et al.(1995)
The Minimal Gene Complement of Mycoplasma genitalium.
Science 270:397-403

\bibitem{label28}
Barrick D. et al.(1994)
Quantitative analysis of ribosome binding sites in E.coli.
Nuc.Acid.Res. 22:1287-1295

\bibitem{label29}
Mehdi H. et al.(1990)
Initiation of translation at CUG, GUG and ACG codons in mammalian
cells.
Gene 91:173-178

\bibitem{label30}
Ding-Fang Yun et al.(1996)
mRNA sequences influencing translation and the selection of AUG
initiator codons in the yeast Saccharomyces cerevisiae.
Mol.Microbiol.19:1225-1239

\bibitem{label31}
Luukkonen BG et al.(1995)
Efficiency of reinitiation of translation on human immunodeficiency
virus type 1 mRNAs is determined by the length of the upstream open
reading frame and by intercistronic distance.
J.Virol. 69:4086-4094

\end{thebibliography}

\end{document}




