

Our question was how the organisms determine translation initiation
sites. We suggested that nucleotide patterns in translation initiation 
site play
important role for the translation initiation.
To investigate it, series of computational experiments have
been conducted to analyze tendencies of nucleotide distributions
around start codons and skipped AUGs for various species and
taxonomical groups. And we have found that there are some remarkable
patterns around start codons for the most of organisms, which is
consistent with the previous research. But we did not find remarkable
patterns around skipped AUGs, which may promote leaky scannings.

Next, we made a hypothesis that in eucaryotes, if two AUGs are located close
to each other, 
ribosomes will confuse to select the appropriate AUG as the start
codons. We investigated it from four approaches. Then the result
showed that there is a tendency to have fewer AUG trinucleotides in
front of start codons. This indicates that AUG trinucleotides just in 
front of start codons are unpreferable for the appropriate selection
of start codons. 
However existing AUG trinucleotides just in front of
start codons do not seem to have much influence on fidelity of the selection of
the appropriate AUGs. 
And the interesting discovery is that higher organisms tend to place
AUG trinucleotides farther from start codons. One interpretation is
that this is because higher organisms do not have much sophisticated
mechanism for the translation initiation. This hypothesis may be
extended to mechanisms of organisms at molecular level in
general. Eucaryotes have much DNAs 
that do not code protein. In other words, genome of higher
organisms are redundant, if non-coding region of DNA does not have
much role. As discussed in information theory, redundancies will
permit errors\footnote{According to the information theory, if the
minimum Hamming distance is more than 
\(2t + 1\), \(t\) errors can be corrected.}. On the other
hand, genomes of procaryotes are less 
redundant. Thus we suggest that for procaryotes, less errors are
permissible and mechanisms must be much sophisticated to prevent errors.

Finally, we have analyzed the distributions of stop codons located
downstream of skipped AUGs and suggested that those stop codons are
presumably playing some important role for leaky scannings.

In this paper, we discovered tendencies that may be involved in
translation initiation. But further analyses from various view points
must be done to understand about the mechanism of translation
initiation. Computer scientific algorithms may be useful to
discover some knowledge about translation initiation from the databases.
And if we can model the translation initiation mechanisms simply,
computer simulation may be helpful to discover some factors 
involved in translation initiation.


% \begin{itemize}
% \item Tendencies involved in translation initiations are discovered.
% \item Further analyses from various point of view must be done to 
% understand completely about translation initiation.
% \item Computer scientific algorithms must be applied to discover rules
% from the databases.
% \end{itemize}