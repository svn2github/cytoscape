
In this paper, analyses of translation initiation sites were conducted
from various points of view to discover tendencies hidden in
translation initiation sites which may explain biological phenomena.

First, comprehensive analyses of consensus patterns surrounding start
codons in various species were conducted to determine translation
initiation signals around start codons. It was found that all
orgarnisms that were investigated have remarkable signals, most of
which are known ones, such as Shine-Dalgarno sequences. However,
several species, such as {\it M.genitalium}, seem to have
alternative signals, suggesting that they are presumably using
alternative mechanisms to initiate translation.

Then characteristics of alternative start codons were investigated,
and found that in some species, alternative start codons were likely
to have weaker Shine-Dalgarno sequences with them. 

And characteristics of base correlations surrounding start codons
were investigated and it was found that in procaryotes, there were
correlations especially between adjacent bases surrounding translation
initiation sites to form translation initiation signals,
whereas there were less correlations to form consensus sequences in
eucaryotes, presumably because procaryotic rRNAs bind mRNA whereas
eucaryotic rRNAs do not at the stage of translation initiation.

Finally, frequencies of ATG triplets around start codons were computed
to investigate how those triplets around start codons are negatively
selected. It turned out that patterns of frequencies of ATG triplets are
different in procaryotes and eucaryotes, presumably due to difference of
translation initiation mechanisms in procaryotes and eucaryotes.
Similar analysis with archaebacteria showed that in some archaebacteria,
ribosome may scan mRNA to find Shine-Dalgarno sequences.

By the comprehensive computer analyses, it was possible to confirm
that tendencies in translation initiation sites follow rules that were
suggested by biological experiments and it was possible to find
exceptions to these suggested rules.

Several biological discoveries on translation initiation were achieved
in this research, and interpretations for them were given. These are
the main original contributions of this paper. By verifying these
experimentally in the future, the contributions will be greater.



%And the interesting discovery is that higher organisms tend to place
%AUG trinucleotides farther from start codons. One interpretation is
%that this is because higher organisms do not have much sophisticated
%mechanism for the translation initiation. This hypothesis may be
%extended to mechanisms of organisms at molecular level in
%general. Eucaryotes have much DNAs 
%that do not code protein. In other words, genome of higher
%organisms are redundant, if non-coding region of DNA does not have
%much role. As discussed in information theory, redundancies will
%permit errors.
%On the other
%hand, genomes of procaryotes are less 
%redundant. Thus we suggest that for procaryotes, less errors are
%permissible and mechanisms must be much sophisticated to prevent errors.

%Finally, we have analyzed the distributions of stop codons located
%downstream of skipped AUGs and suggested that those stop codons are
%presumably playing some important role for leaky scannings.

%In this paper, we discovered tendencies that may be involved in
%translation initiation. But further analyses from various view points
%must be done to understand about the mechanism of translation
%initiation. Computer scientific algorithms may be useful to
%discover some knowledge about translation initiation from the databases.
%And if we can model the translation initiation mechanisms simply,
%computer simulation may be helpful to discover some factors 
%involved in translation initiation.


